\documentclass[a4paper, 12pt]{article}

%% Language and font encodings. This says how to do hyphenation on end of lines.
\usepackage[english]{babel}
\usepackage[utf8x]{inputenc}
\usepackage[T1]{fontenc}
\usepackage{algorithm,algorithmic}
\usepackage{authblk}

%% Sets page size and margins. You can edit this to your liking
\usepackage[top=1.3cm, bottom=2.0cm, outer=2.5cm, inner=2.5cm, heightrounded,
marginparwidth=1.5cm, marginparsep=0.4cm, margin=2.5cm]{geometry}

%% Useful packages
\usepackage{graphicx} %allows you to use jpg or png images. PDF is still recommended
\usepackage[colorlinks=False]{hyperref} % add links inside PDF files
\usepackage{amsmath}  % Math fonts
\usepackage{amsfonts} %
\usepackage{amssymb}  %
\usepackage{mathscinet,amsthm,amssymb,subcaption,graphicx,epstopdf,multicol}
\usepackage{color}

%Chris packages%%%%%%%%%%%%%%%%%%%%%%%%%%%%%%%%%%%%%%
\usepackage[version=3]{mhchem}
\graphicspath{{./figs/}}
% \usepackage[%
% figurewithin=section,
% tablewithin=section
% ]{caption}
%%%%%%%%%%%%%%%%%%%%%%%%%%%%%%%%%%%%%%%%%%%%%%%

%% Citation package
%\usepackage[authoryear]{natbib}
\bibliographystyle{plain}
%\setcitestyle{authoryear,open={(},close={)}}

%% Sort out spacing in tables so that they look nice
{\renewcommand{\arraystretch}{1.1}}

{\newcommand{\comment}[1]{{\color{red} \bf{#1}}}

\newcommand{\beginsupplement}{%
        \setcounter{table}{0}
        \renewcommand{\thetable}{S\arabic{table}}%
        \setcounter{figure}{0}
        \renewcommand{\thefigure}{S\arabic{figure}}%
        \setcounter{section}{0}
        \renewcommand{\thesection}{S\arabic{section}}%
        \setcounter{equation}{0}
        \renewcommand{\theequation}{S\arabic{equation}}%
        
        
     }

\title{Integration of experimenter-controlled heuristic and automated data optimization methods in the parametrization of three unresolved two-electron surface-confined \ce{[PMo12O40]3-} reduction processes by AC voltammetry}

\author[1,*]{Martin Robinson}
\author[2]{Kontad Ounnunkad}
\author[2]{Jie Zhang}
\author[1,*]{David Gavaghan}
\author[2,*]{Alan Bond}

\affil[1]{Department of Computer Science, University of Oxford, Wolfson Building, Parks Road, Oxford, OX1 3QD, United Kingdom.}
\affil[2]{School of Chemistry, Monash University, Clayton, Vic. 3800, Australia.}

\affil[*]{Corresponding authors: martin.robinson@cs.ox.ac.uk, david.gavaghan@cs.ox.ac.uk, alan.bond@monash.edu.au}

\date{\today}

\begin{document}


\maketitle

\begin{abstract}
The thermodynamic and electrode kinetic parameters that describe the three 
    unresolved proton-coupled two-electron transfer processes associated with 
    the overall six electron reduction of surface-confined Keggin-type 
    phosphomolybdate, \ce{[PMo12O40]3-} adsorbed onto glassy carbon (GC) 
    electrode has been elucidated in 1M H2SO4.  Modeling of this problem 
    requires the introduction of over 30 parameters, although with 
    implementation of sensitivity analysis and other strategies this may be 
    reduced to about 15 when Fourier transformed large amplitude alternating 
    current voltammetry (FTACV) and intelligent forms of data analysis are 
    introduced. Heuristic (experimenter based tedious trial and error method) 
    and automated computationally intensive date optimization approaches are 
    combined in this exceptionally extensive parameter estimation exercise.  
    However, obtaining a unique solution in this multi-parameter experiment- 
    theory data fitting exercise is exceptionally challenging and is achieved by 
    a hybrid of the heuristic and data optimization methods. The strategies used 
    to achieve a chemically credible set of parameters in a voltammetry data 
    fitting exercise of this complex kind are presented in detail. In the final 
    analysis six reversible potentials, six electron transfer rate constants, 
    the double layer capacitance, uncompensated resistance, surface coverage and 
    some AC experimental parameters are reported, with others present in the 
    model being unobtainable for reasons that are provided.

  
\end{abstract}

\section{Introduction}

Voltammetric theory for very complex electrode processes comprising an extensive 
series of coupled heterogeneous electron transfer steps and homogeneous chemical 
reactions is now very well established. Generation of theoretical data derived 
from a designated model is known as the forward problem. However, for a very 
complex electrode process, obtaining a large  number of unknown parameters that 
have to be deduced by comparison of experimental and theoretical data, in what 
is termed the inverse problem, often still remains unmanageable with respect to 
obtaining a complete and unique solution. In essence, solving the inverse 
problem requires capturing substantial amounts of very high quality experimental 
data and repetitively comparing with theoretical data deduced from a model until 
acceptable agreement is achieved. When  even the simplest possible  process in 
which  an oxidized electroactive species (Ox) is reduced to its reduced form 
(Red) as  summarized in equation 1,   and modelling is undertaken assuming  the 
Butler-Volmer relationship applies and mass transport occurs solely by planar 
diffusion, there is likely to be a minimum of 5 parameters that have to be 
estimated; viz formal reversible potential ($E_0$), heterogeneous charge 
transfer rate constant ($k_0$), charge transfer coefficient ($\alpha$), , double 
layer capacitance ($C_{dl}$) and uncompensated resistance ($R_u$), assuming 
diffusion coefficients ($D_{ox}$ and $D_{red}$) and other relevant parameter 
values are known from independent measurements. If chemical steps are coupled to 
multi-electron transfer then in excess of ten unknown parameters will almost 
certainly need to be quantified (ref).  When addressing a problem of this or 
greater complexity, it may even have to be concluded that   that full 
parameterization is impossible to achieve when experimental error and model 
uncertainty are taken into account.

\begin{align} \label{eq:reaction}
Ox + e^- \cee{&<=>[E_0,k_0,\alpha, C_{dl}, R_u]} Red,
\end{align}

The  forward process of predicting the theory using a proposed model, which was 
once demanding when computer coding of each step was required in each study, can 
now be achieved routinely with user friendly, commercially available software 
packages such as DigiSim or DigiElch or by using freeware that can be downloaded 
from the web such as MECSIM (ref). Now it is the inverse problem of deciding 
which model and combination of parameters best describes the experimental data 
and how good is the fit that usually presents a daunting challenge. Typically, 
the experimentalist who collected the data may elect to “guess” the model that 
is applicable and rely on experience to fit the data by essentially trial and 
error procedures in what almost invariably becomes an extensive series of very 
tedious theory-experiment caparisons. In this heuristic approach, the 
experimentalist decides empirically when an acceptably good fit of data has been 
achieved and then provides a report of the mechanism and parameter values that 
fit the “guessed” mechanism. As an alternative, multi-parameter fitting aided by 
computationally efficient data optimization or more sophisticated approaches 
based on Bayesian inference or other statistical   methods are now available to 
assist with solving the inverse problem.  (refs) Data optimization methodology, 
underpinned by statistics and facilitated by high speed computing, has been 
developed to support many branches of science. Now, as in simulation packages 
developed for the forward problem there are an extensive range of software 
packages available to support complex theory-experiment inverses data 
optimization exercises. Nevertheless, in voltammetry, problems requiring in 
excess of 10 parameters to be estimated are rarely attempted. In particular, if 
the data set available is inadequate as often is the case when using DC cyclic 
voltammetry (inadequate number data points or range of scan rates etc.), then 
the significance of the outcome of a multi-parameter fitting exercise is likely 
to remain equivocal.

In recent work, the Oxford and Monash University Groups have been developing 
protocols to address the issues arising when attempting to parameterize 
increasingly complex mechanisms. In summary, very large data sets containing 
extractable variable time (frequency) domain information are now collected at 
high resolution using instrumentation having 18 bit DAC and ADC converters 
(refs). The waveforms are based on employing a large amplitude periodic waveform 
superimposed onto a cyclic DC potential ramp, so that features related to 
electrochemical impedance spectroscopy and DC cyclic voltammetry are 
simultaneously available as well as additional ones (Refs). Experiment-theory 
comparisons have then been undertaken at levels ranging from fully heuristic to 
multi-parameter fitting in attempts to uniquely define the thermodynamic, 
kinetic, mass transport, capacitance and resistance  related parameters that are 
included in the model used to generate the simulated data. This modelling 
approach uses parameters that have a direct relationship to the physical 
chemistry associated with the reaction mechanisms  unlike the use of equivalent  
circuit models employed traditionally employed in electrochemical impedance 
spectroscopy (EIS), although both approaches are of course mathematically 
interchangeable. 


In the present study, we have taken advantage of access to ever expanding 
computing power as well as software available from many sources to quantify a 
complex mechanism which even in a simplified form contained 15 unknown 
parameters in the model. These parameters were quantified initially by both 
fully heuristic and fully automated data optimization methods. However, a hybrid 
approach in which automated data optimization strategies is informed by 
knowledge gained from the heuristic method of data analysis has been found to 
provide the best fit of theoretical to experimental data. 

Polyoxometalates (POMs) which are of interest in this paper have been widely 
employed in chemistry in diverse fields such as electrocatalysis and 
photocatalysis \cite{symes2013decoupling,lee2012detailed}, sensors 
\cite{wei2014phosphomolybdic,wang2014novel} and capacitors 
\cite{bajwa2013multilayer} and are of interest in many branches of science and 
technology. Since many practical devices based on POMs exploit their extensive 
redox capacity, detailed studies of their electrochemistry are needed to 
facilitate their development.  Due to the widespread importance of POMs but  
inherent complexity of their multistep electron transfer pathways, as expected, 
many qualitative electrochemical studies POMS  by voltammetric methods are 
available,  with quantitative ones almost invariably being  confined to just the 
initial one or two  electron transfer processes. The electrochemistry quantified 
in this study is the reduction of the surface confined polyoxometalate 
\ce{[PMo12O40]3-} This inorganic cluster contains 12 molybdenum atoms in 
oxidation state VI that can be reduced to mixed valent forms in multi-electron 
steps to give highly charged and very basic mixed-valent products that 
facilitate coupling of electron and proton transfer reactions that can occur in 
many combinations. 

The () anion, which is of interest in this paper, in the structural sense is a 
Kegging-type POM (add structural figure) whose voltammetry at solid electrodes 
and polarography at the dropping mercury electrode has been extensively reported 
when dissolved in aqueous electrolyte media, molecular solvents containing 
supporting electrolytes or ionic liquids. In acidic aqueous media,   is known to 
spontaneously adsorb onto electrode surfaces such as glassy carbon 
\cite{choi2009adsorption}, gold \cite{choi2009adsorption}, silver 
\cite{choi2009adsorption}, and reduced graphene oxide 
\cite{ding2014phosphomolybdate}.  DC and AC cyclic voltammograms shown in in 
Figure 2 (to be added) were obtained at a glassy carbon (GC) electrode in 1 M 
H2SO. The three  surface confined reduction processes displayed in these 
voltammograms under  the acidic conditions relevant to this study  and 
designated as Processes I, II and III are each overall  two electron-two proton 
coupled process   that can be represented by  equations 1-3 
\cite{chen2013fabrication}.  

\begin{align}
    \cee{[PMo12O40]3- + 2e- + 2H+ &<=>[E_0,k_0,\alpha, C_{dl}, R_u] 
    H2[PMo12O40]3-} \qquad \text{(Process I)} \\
    \cee{H2[PMo12O40]3- + 2e- + 2H+ &<=>[E_0,k_0,\alpha, C_{dl}, R_u] 
    H4[PMo12O40]3-} \qquad \text{(Process II)} \\
    \cee{H4[PMo12O40]3- + 2e- + 2H+ &<=>[E_0,k_0,\alpha, C_{dl}, R_u] 
    H6[PMo12O40]3-} \qquad \text{(Process III)}
\end{align}


The peak potentials and are located   at about 350 (Process I), 225 (Process 
II), and 10 (Process III) mV vs Ag/AgCl in 1 M H2SO4 and depend on acid 
concentration (shift of about 60mV per unit change in H2SO4 concentration change 
over the range 0.01 to 1.0M) as expected if   a net two- proton transfer 
reaction accompanies a net two-electron transfer reaction. However, extensive 
reduction to give Process IV (not shown) leads to rapid dissolution of solid 
while use of lower acidities facilitates dissolution. Furthermore, describing 
the fully oxidized as completely unprotonated in 1 M H2SO4 is unlikely to be 
correct.  


Points for noting gained from perusal of Figure 2 and which should be 
accommodated in simulation-experiment comparisons are that the first two 
processes partially overlap, Process II has a larger peak current magnitude than 
either Processes I or III and the shapes and current magnitudes are not equal 
for all three processes. Importantly, for each of the three reduction steps it 
has been proposed [9-14] that two unresolved reversible one electron transfer 
steps of purely adsorbed material occur as in equations 5 and 6 (5)
 									 (6)		

where Oxsurf and Redsurf are the fully oxidized formally [PMo12O40]3- (written 
for convenience in the non-protonated form) and 2 electron reduced formally 
protonated [PMo12O40]5-  species bound to the electrode surface  while Isurf is 
an intermediate or half-way reduced surface confined  protonated [PMo12O40]4- 
species. Even the simplest possible level of data analysis with all six electron 
transfer steps treated as reversible and all other parameter values known 
requires the determination of six reversible formal potentials (E0 values).  
Inclusion of the electrode kinetics for requires the addition of six electron 
transfer rate constants  ($k_0$  values) along with six charge transfer 
coefficients ($\alpha$ values) if the Butler-Volmer relationship is used to 
model the electron transfer processes. Additionally, $C_{dl}$, $R_u$, and the 
adsorption isotherm need to be modelled as in principle do the thermodynamics 
and kinetics of the chemical (acid-base)   reactions coupled to electron 
transfer. The existence of unresolved overall 2 electron processes precludes any 
simple interpretation of the thermodynamics and electrode kinetics and there are 
well in excess of thirty parameters present in the full POM reduction modeling 
exercise. As noted above, even at this level of complexity, all parameters could 
be included in the forward problem using MECSim, DigiElch or other software 
packages. However, it is not likely to be realistic to simply undertake a 30 or 
more parameter determination exercise using a manual heuristic or even an 
automated data so the optimization experiment –theory fitting exercise and 
expect to obtain a chemically and statically reasonable unique value for each 
one. In a practical sense it will be shown in this study that the best 
opportunity to approach this goal is by integration of both heuristic and 
automated data optimization approaches to achieve   a high level of agreement 
between experiment and theory for about 15 parameters that are shown to 
significantly influence the voltammetry.

Finally, in this Introduction it is emphasized that the data set obtained 
experimentally has to be sufficient in quality and quantity to justify the 
conclusions reached in a parameterization exercise and capable of being mimicked 
by the simulations. AC voltammetry in square wave, sinusoidal or other forms as 
well as EIS provide significant advantages over traditional DC methods in 
quantitative studies of electrode processes [15-16]. If large amplitude AC 
signals are employed, the significantly amplified higher-order AC harmonic 
components become available that are virtually free of background capacitance 
current and are also highly sensitive to electrode kinetics. The ability to 
resolve the aperiodic DC, fundamental and higher order harmonics is crucial in 
heuristic forms of data analysis. For simple problems, comparison of 
experimental data with the numerical simulations of appropriate models have 
provided  good estimates of $E_0$,  $k_0$, $\alpha$, $R_u$ and $C_{dl}$ from a 
single experiment, using  both heuristic and data optimization forms of 
experiment -theory comparison. Large amplitude AC voltammetry also has been 
applied to the determination of the electrode kinetics of surface-confined 
enzymes. In some cases, their behavior  mimics that   with \ce{[PMo12O40]3-}   
in the sense  that  acid-base chemistry   can be coupled with an unresolved 2 
electron transfer process to provide a problem  akin to a combination of 
equations 2, 5 and 6. In this study, experience acquired in parameterization of 
these simpler systems (refs) is crucial in addressing the vastly more complex 
problem of parameterization of the six electron reduction of surface confined 
\ce{[PMo12O40]3-}.

\section{Experimental}

\subsection{Chemicals and Reagents}


Sulfuric acid (\ce{H2SO4}, Merck,) and phosphomolybdic acid (\ce{H3PMo12O40}, 
Sigma-Aldrich, ≥ 99.99\%), were used as received. Water employed for preparation 
of all aqueous solutions was obtained from a Milli-Q water purification system.

\subsection{Instrumentation and procedures}


AC voltammetric measurements were undertaken using in-house instrumentation with 
an applied sine wave of amplitude ($\Delta E$) of 20 mV and frequency ($f$) of 
9.02 or 60.05 Hz being superimposed onto the DC cyclic ramp which had a scan 
rate ($v$).   DC voltammetric data were also obtained with this instrument. A 
conventional three-electrode potentiated system was used to collect experimental 
data at 25°C.  GC (diameter = 3.0 mm, CH Instruments, USA) was used as the 
working electrode,
\ce{Ag/AgCl} (3M KCl) as the reference electrode and platinum wire as the auxiliary electrode with 1.0M \ce{H2SO4} being the supporting electrolyte.

\subsection{Simulations and data analysis}


Simulations of total current AC voltammograms used in heuristic forms of data 
analysis were carried out with MECSim (Monash Electrochemistry Simulator) 
software. Modelling was undertaken with code specifically written for use in the 
automated data optimization exercises. Details of the model and parametrization 
employed are provided in the Results and Discussion section.  In Fourier 
transformed  AC voltammetry (FTACV) the experimental and simulated total current 
time domain data were converted to the frequency domain to generate the power 
spectrum [39,40]. Frequencies corresponding to the AC harmonics and the 
aperiodic DC component were then selected from the power spectrum and then 
subjected to band filtering and inverse Fourier transformation to obtain 
resolved DC and AC components as a function of time. Ru was determined 
experimentally from the RuCdl time constant at potentials where no faradaic 
current was present.

\section{Results and Discussion}

\subsection{The Model}


To simulate the total current AC voltammetric data, a mathematical model has to 
be employed. To mimic the 3 surface confined processes summarized in equations 2 
to 4, a series of well-known relationships were employed in combination to give 
a total model as follows:

1. Electron transfer model: the Butler-Volmer relationship was used for each of 
the six electron transfer steps which requires the introduction of six $E_0$, 
$k_0$ and $\alpha$ parameters.  Alternately, the mathematically more 
sophisticated Marcus-Hush relationship could have been used. Neither 
thermodynamic nor kinetic dispersion are included in the electron transfer 
model.

2. Acid-base chemistry: proton transfer reactions coupled to electron transfer 
were assumed to be diffusion controlled and hence reversible and subject to a 
solely thermodynamic description as is E0. On this basis, the unknown acid (pKa) 
equilibrium constants have been combined with the E0 values so effectively only 
six reversible potential (Erev) thermodynamically relevant parameters with their 
six krev and $\alpha$rev values need to be parameterized. However, this means 
that the individual $E_0$, $k_0$, $\alpha$, and pKa values remain unknown.  3.  
Adsorption isotherm: the Langmuir model was used, which means that a surface 
coverage parameter is needed. Other isotherms are available such as Frumkin or 
Temkin.

4. Uncompensated resistance: Ohms law used (Ru parameter introduced).

5. Double layer capacitance: Modeling undertaken assumes a simple RuCdl time 
constant   applies at each potential and that $C_dl$ is independent of potential 
(Cdl parameter needed). In fact Cdl depends on potential

6. Numerous other parameters are included in the model (see theory below), but 
most are known such as AC amplitude DC scan rate, electrode area, temperature 
etc. However, as a check on fidelity of experimental data, AC frequency and 
phase angle are treated as parameters to be estimated in the automated but not 
heuristic data optimization approach to ensure absence of instrumental 
artefacts. 

In summary, the   model chosen for simulation of AC voltammograms   requires a 
total of 17 parameters to be estimated by automated data optimization and 15 
heuristically under circumstances where there are imperfections and 
uncertainties in each aspect of the model as noted above 

Martin to add model details in terms of A, B, C, D, E, and F species and math’s

\subsection{Heuristic Method of Parameter Estimation}


The heuristic or exclusively experimenter based trial and error method to be 
tractable requires simplifying the problem and obtaining initial estimates of 
parameters in a sensible stepwise fashion by relying on experience gained with 
simpler problems. In this exercise, the solution to the forward problem was 
obtained from the simulation package MECSim with an initial guess made for each 
parameter. Simulations with new sets of parameters were then iteratively changed 
in the direction dictated by the experience of the experimenter until this 
individual decided an acceptably “good fit” to data had been achieved. Of course 
a different experimentalist may decide that the “good fit” is achieved with a 
different set of parameters!


The initial scrutiny of data involved visual interrogation of   20mV amplitude 
FTAC data sets at 9 and 60 Hz (Figs x and y), their power spectra and the first 
4 AC harmonics (Figs e, f) and a dc cyclic voltammogram (figure a). This 
overview allowed initial guesses or estimates to be made for some parameters 
along with some deductions as to the validity of the model as follows:

(a)  It was noted in the fundamental AC harmonic that a potential (time) region exists prior to onset of faradaic current for process I that is purely capacitive.  This region was analyzed in terms of the RuCdl time constant to give an estimate of Ru as 50 +/- 5 Ohm. 

(b) By matching experiment and theory  for the same potential (time) region in 
the fundamental   harmonic that is devoid of faradaic current a described in (a) 
and effectively use of equation x (to be added), Cdl was estimated to be 7 +/- 1 
(TODO: units) F cm-2. However, the sloping baseline in  the DC cyclic 
voltammogram and mismatch of positive and negative potential regions of 
fundamental harmonic devoid of faradaic current, and indeed the aperiodic DC 
component of FTACV data, it is clear that Cdl  potential dependent, rather than 
potential independent, as assumed in the model. Thus, treating Cdl as a 
potential independent parameter in an experiment versus theory exercise will 
represent an imperfect approximation.  More complex potential dependent 
estimates of Cdl could have been included in the model. However, perusal of 
third and fourth AC harmonics (Figure x) reveals no background current in these 
responses, implying that the electrode kinetics can be estimated from these data 
that are devoid if capacitance current.  Consequently, moving to the more 
complex capacitance model was not regarded as necessary, particularly under 
conditions where the Ohmic drop or ItotalRu is small.

(c) The surface coverage was found to be $\Gamma$ = 50 +/- 5 pmole cm-2 based on 
integration of DC current time-data to give the charge associated with reduction 
of the POM and use of Faraday’s Law (equation y to be added). The charge used in 
the estimation of $\Gamma$ was calculated from the sum of that derived from 
processes I and II (n=4) or process   III (n = 2) after   background 
(capacitance current) correction of a DC linear sweep voltammogram (ref)

(d) Since all three processes are known to be overall two electron, are sharp 
and give a series of well-defined higher order AC components, the k0 values were 
assumed to be large. The initial guess was that all are sufficiently large and 
that with a low frequency of 9Hz that all one electron processes could initially 
be assumed to be reversible.  Furthermore, the two Erev values associated with 
each of processes I, II and III are clearly not well separated in the direction 
that the second electron transfer step is more negative than the second.  Thus, 
the reversible potentials of the two  overlapping  steps in all cases were 
concluded to be either almost the same  or the second electron transfer step was 
less negative  than the first (crossed over potentials because of the 
thermodynamic contribution of protonation step coupled to electron transfer as 
noted above) To obtain more detail, Figure 2 was generated to provide simulated 
fundamental and third harmonic AC  data where the separation of the two Erev 
values in process was II fixed at zero (largest current magnitude ) while  
processes  II and III had variable separations in the Erev ($\Delta E_{rev}$) 
values from 0 to 55 mV with all $k_0$ and $\alpha$  values being set at 1 x 10 8  
s-1  and 0.50 respectively to mimic a reversible process  and other parameters 
set at estimates approximating those deduced in (a), (b) and (c) above.  Greater 
crossover potentials   than 55 mV were deemed unreasonable. By eye, and with 
some fine tuning, $\Delta E_{rev} = 30$, 0 and 27  for process I, II and III 
respectively gave match of theory and experiment for the 9Hz low frequency data 
set. To confirm that the assumption of reversibility is valid and what this 
means, Figure 3  was generated with  simulated fundamental and third harmonic 
response for the pairs of k0 values always the same but ranging from 1.0 x103 to 
10 8  s-1  with $\alpha$ being set at 0.50 and $\Delta E_{rev} = 33$, 0 and 45.  
At the low frequency of 9 Hz, all three processes are insensitive to k0 
electrode kinetics in excess of 102 s-1. Thus it was concluded from this trial 
and error analysis that $\Delta E_{rev} = 30$, 0 and 27 were appropriate for 
processes I, I and III which translates to Erev values of 368, 338, 227, 227 11 
and -16 mV Ag/AgCl and that some if not all k0 values are larger than about 102 
s-1.   

(e) Examination of the higher frequency 60 Hz data set implied that the 
electrode kinetics could possibly be determined from theory experiment 
comparisons of the higher order AC harmonics However, the only reasonable trial 
and error heuristic approach available was to set the k0 value for each pair of 
processes to be equal with $\alpha = 0.50$, assume Erev values and other 
parameters are known from 9 Hz data or can be re-estimated as for 9 Hz data. Now 
only  pairs of ko values are the along with the surface coverage scaling factor 
$\Gamma$ remain as unknowns. After much tedious trial and error examining many 
simulations, the reasonably good match of theory and experimental data shown in 
figure x was obtained for the total current data and in figure y for AC 
harmonics 1 to 4 when the parameters summarized in Table 1 and in captions to 
these Figures are used.  Never the less, the significance of the absolute ko 
values deduced from this exercise and included in Table 1 remains unclear, apart 
from implying  that fast electrode kinetics  approaching the reversible limit is 
evident for each of processes I, II and III

Since the fit obtained heuristically is imperfect and to make tractable had to 
assume that pairs of k0 values are equal, it now needs to be established as what 
outcome could be achieved   via multi-parameter data optimization and how this 
exercises should be implemented since almost certainly simply seeking to obtain 
15 or more unknown parameters from a single multi-parameter fit would be futile.  
Indeed, it will emerge from this exercise that a hybrid approach based using 
knowledge obtained above heuristically, supported by computationally efficient 
data optimization are needed to obtain chemically sensible outcomes.


\subsection{Mathematical Modelling}


The details of the mathematical and computational modelling approach that we 
have taken in this paper were given previously in \cite{Morrisetal,gavaghan 
2017}. In summary, our chosen experimental system is modelled as a 
quasi-reversible reaction 

\begin{align} \label{reaction}
A \cee{&<=>[E_0,k_0,\alpha]} B + e^-, 
\end{align}

where species $A$ and $B$ are in solution, and $E_0$, $k_0$, and $\alpha$  are 
the reversible formal potential, standard heterogeneous charge transfer rate 
constant at $E_0$ and the charge transfer coefficient, respectively, and we 
assume that the Butler-Volmer formalism is used to describe the electron 
transfer process 
\cite{BardFaulkner,brett1993principles,pletcher2001instrumental}. We assume that 
both convection and migration can be neglected since we are using a macrodisk 
stationary electrode, and an excess of supporting electrolyte respectively.  
Since we assume equal diffusion coefficients for each species $A$ and $B$ ($D_A 
= D_B = D$) we need solve for the concentration of only one of the species  
(i.e. the concentrations $c_A$, $c_B$ of species $A$ and $B$ satisfy $c_A   = 
c_{\infty} - c_B $, where $c_\infty$ is the bulk concentration of species $A$) 
and we choose to solve for species $A$. We can then use Fick's second law to 
model the variation with time of species $A$ via

\begin{align}
\frac{\partial c_A}{\partial t} &= D \frac{\partial^2 c_A}{\partial x^2}, \label{diffeqtn}
\end{align}

where $x$ is distance from the electrode surface and $t$ is time. The initial 
and boundary conditions are 

\begin{align}
c_A(x,0) &= c_{\infty} \nonumber \\
c_A &\rightarrow c_{\infty},  \quad \text{as} \quad x \rightarrow \infty,\quad t>0. \label{icsandbcs}
\end{align}

At the electrode surface, $x=0$, for $t>0$, we have the conservation and flux conditions

\begin{align}
D \frac{\partial c_A}{\partial x}=\frac{I_f}{FS}, \label{fluxatelectrode}
\end{align}

along with the Butler-Volmer condition

\begin{align}
D \frac{\partial c_A}{\partial x} = \text{ }&k_0 \left[c_A\exp\left((1-\alpha )\frac{F}{RT}(E_{\mbox{\tiny eff}}(t)-E_0)\right)\right.
\left.-(1-c_A)\exp\left(-\alpha\frac{F}{RT}(E_{\mbox{\tiny eff}}(t)-E_0)\right)\right]. \label{BVeqtn}
\end{align}

Here, $I_f$ is the faradaic current, $S$ is the electrode area, and 
$E_{\mbox{\tiny eff}}(t)$ is the {\em effective} applied potential (defined 
below).

We complete the model by defining $E_{\mbox{\tiny{app}}}(t)$ to be the applied 
potential, then for the case of AC voltammetry ramp we have

\begin{align}
E_{\mbox{\tiny{app}}}(t) =E_{\mbox{\tiny{start}}} \pm (v t + \Delta E \sin {(\omega t)}), \qquad 0   \label{EAC}
\end{align}

where $v$ is the sweep rate,  $E_{\mbox{\tiny{start}}}$ is the initial 
potential, $t_{\max}$ is the end time of the experiment, and for the AC case, 
$\omega$ is the radial frequency and  $\Delta E$ the amplitude of the sine wave.  
The {\em effective} applied potential can now be defined as 

\begin{align}
E_{\mbox{\tiny{eff}}}(t) = E_{\mbox{\tiny{app}}}(t) - E_{\mbox{\tiny{drop}}} = E_{\mbox{\tiny{app}}}(t) - I_{\mbox{\tiny{tot}}} R_u \nonumber
\end{align}

where $E_{\mbox{\tiny{drop}}}$ models the effect of uncompensated resistance, 
$R_u$. $I_{\mbox{\tiny{tot}}}$ is the total (measured) current, and combines the 
faradaic current and the background capacitive current, $I_c$, which can be 
modelled as   

\begin{align}
I_c &= C_{dl}\frac{dE_{\mbox{\tiny{eff}}}}{dt}, \label{Ic}
\end{align}

where $C_{dl}$ is the double layer capacitance (assumed constant in this work), and then 

\begin{align}
I_{\mbox{\tiny{tot}}} = I_f + I_c. \label{Itot}
\end{align}

As described in the introduction in Eqs. \ref{Lambda}, for the case of DC 
voltammetric methods the values of $k_0$, $D$ and $v$ determine whether the 
reaction is fully-reversible (large $\Lambda$), quasi-reversible (intermediate 
$\Lambda$), or irreversible (small $\Lambda$). 


Equations \ref{diffeqtn}--\ref{Itot} are non-dimensionalised as described in the 
supplementary information. The resulting non-dimensional system of equations is 
solved using an implicit finite difference method with an exponentially 
expanding grid (as we have described in full detail previously - see 
\cite{Sheretal}). Having derived the governing partial differential equations 
and boundary conditions for this simplest problem of the reaction mechanism in 
Eq. \ref{reaction} we can see mathematically that the mechanism is governed by 
five parameters $(k_0, E_0, \alpha, R_u, C_{dl})$, and we will collectively 
denote these parameters by the vector $ \boldsymbol\theta$. The {\em inverse} 
problem that we wish to solve can be defined as finding the best possible 
approximation to $\boldsymbol\theta$ given our measured experimental output 
trace of the current $I_{\mbox{\small{\tiny{tot}}}}^{\mbox{\tiny{data}}}$, 
versus potential, together with an estimate of the effect of experimental noise.  
This will be achieved by recovering the marginal posterior distributions of each 
parameter that are induced by the experimental noise.  



\end{document}
