\documentclass[a4paper, 12pt]{article}

%% Language and font encodings. This says how to do hyphenation on end of lines.
\usepackage[english]{babel}
\usepackage[utf8x]{inputenc}
\usepackage[T1]{fontenc}
\usepackage{algorithm,algorithmic}
\usepackage{authblk}

%% Sets page size and margins. You can edit this to your liking
\usepackage[top=1.3cm, bottom=2.0cm, outer=2.5cm, inner=2.5cm, heightrounded,
marginparwidth=1.5cm, marginparsep=0.4cm, margin=2.5cm]{geometry}

%% Useful packages
\usepackage{graphicx} %allows you to use jpg or png images. PDF is still recommended
\usepackage[colorlinks=true,citecolor=blue]{hyperref} % add links inside PDF 
\usepackage{amsmath}  % Math fonts
\usepackage{amsfonts} %
\usepackage{amssymb}  %
\usepackage{mathscinet,amsthm,amssymb,subcaption,graphicx,epstopdf,multicol,bm}
\usepackage{color}

%Chris packages%%%%%%%%%%%%%%%%%%%%%%%%%%%%%%%%%%%%%%
\usepackage[version=3]{mhchem}
\graphicspath{{./figs/}}
% \usepackage[%
% figurewithin=section,
% tablewithin=section
% ]{caption}
%%%%%%%%%%%%%%%%%%%%%%%%%%%%%%%%%%%%%%%%%%%%%%%

%% Citation package
%\usepackage[authoryear]{natbib}
\bibliographystyle{plain}
%\setcitestyle{authoryear,open={(},close={)}}

%% Sort out spacing in tables so that they look nice
{\renewcommand{\arraystretch}{1.1}}

{\newcommand{\comment}[1]{{\color{red} \bf{#1}}}

\newcommand{\beginsupplement}{%
        \setcounter{table}{0}
        \renewcommand{\thetable}{S\arabic{table}}%
        \setcounter{figure}{0}
        \renewcommand{\thefigure}{S\arabic{figure}}%
        \setcounter{section}{0}
        \renewcommand{\thesection}{S\arabic{section}}%
        \setcounter{equation}{0}
        \renewcommand{\theequation}{S\arabic{equation}}%
        
        
     }

\title{Integration of experimenter-controlled heuristic and automated data optimization methods in the parametrization of three unresolved two-electron surface-confined \ce{[PMo12O40]3-} reduction processes by AC voltammetry}

\author[1,*]{Martin Robinson}
\author[2]{Kontad Ounnunkad}
\author[2]{Jie Zhang}
\author[1,*]{David Gavaghan}
\author[2,*]{Alan Bond}

\affil[1]{Department of Computer Science, University of Oxford, Wolfson Building, Parks Road, Oxford, OX1 3QD, United Kingdom.}
\affil[2]{School of Chemistry, Monash University, Clayton, Vic. 3800, Australia.}

\affil[*]{Corresponding authors: martin.robinson@cs.ox.ac.uk, david.gavaghan@cs.ox.ac.uk, alan.bond@monash.edu.au}

\date{\today}

\begin{document}


\maketitle

\begin{abstract}
The thermodynamic and electrode kinetic parameters that describe the three 
    unresolved proton-coupled two-electron transfer processes associated with 
    the overall six electron reduction of surface-confined Keggin-type 
    phosphomolybdate, \ce{[PMo12O40]3-} adsorbed onto glassy carbon (GC) 
    electrode has been elucidated in 1M H2SO4.  Modeling of this problem 
    requires the introduction of over 30 parameters, although with 
    implementation of sensitivity analysis and other strategies this may be 
    reduced to about 15 when Fourier transformed large amplitude alternating 
    current voltammetry (FTACV) and intelligent forms of data analysis are 
    introduced. Heuristic (experimenter based tedious trial and error method) 
    and automated computationally intensive date optimization approaches are 
    combined in this exceptionally extensive parameter estimation exercise.  
    However, obtaining a unique solution in this multi-parameter experiment- 
    theory data fitting exercise is exceptionally challenging and is achieved by 
    a hybrid of the heuristic and data optimization methods. The strategies used 
    to achieve a chemically credible set of parameters in a voltammetry data 
    fitting exercise of this complex kind are presented in detail. In the final 
    analysis six reversible potentials, six electron transfer rate constants, 
    the double layer capacitance, uncompensated resistance, surface coverage and 
    some AC experimental parameters are reported, with others present in the 
    model being unobtainable for reasons that are provided.

  
\end{abstract}

\section{Introduction}

Voltammetric theory for very complex electrode processes comprising an extensive 
series of coupled heterogeneous electron transfer steps and homogeneous chemical 
reactions is now very well established. Generation of theoretical data derived 
from a designated model is known as the forward problem. However, for a very 
complex electrode process, obtaining a large  number of unknown parameters that 
have to be deduced by comparison of experimental and theoretical data, in what 
is termed the inverse problem, often still remains unmanageable with respect to 
obtaining a complete and unique solution. In essence, solving the inverse 
problem requires capturing substantial amounts of very high quality experimental 
data and repetitively comparing with theoretical data deduced from a model until 
acceptable agreement is achieved. When  even the simplest possible  process in 
which  an oxidized electroactive species (Ox) is reduced to its reduced form 
(Red) as  summarized in equation 1,   and modelling is undertaken assuming  the 
Butler-Volmer relationship applies and mass transport occurs solely by planar 
diffusion, there is likely to be a minimum of 5 parameters that have to be 
estimated; viz formal reversible potential ($E^0$), heterogeneous charge 
transfer rate constant ($k^0$), charge transfer coefficient ($\alpha$), , double 
layer capacitance ($C_{dl}$) and uncompensated resistance ($R_u$), assuming 
diffusion coefficients ($D_{ox}$ and $D_{red}$) and other relevant parameter 
values are known from independent measurements. If chemical steps are coupled to 
multi-electron transfer then in excess of ten unknown parameters will almost 
certainly need to be quantified (ref).  When addressing a problem of this or 
greater complexity, it may even have to be concluded that   that full 
parameterization is impossible to achieve when experimental error and model 
uncertainty are taken into account.

\begin{align} \label{eq:reaction}
Ox + e^- \cee{&<=>[E^0,k^0,\alpha, C_{dl}, R_u]} Red,
\end{align}

The  forward process of predicting the theory using a proposed model, which was 
once demanding when computer coding of each step was required in each study, can 
now be achieved routinely with user friendly, commercially available software 
packages such as DigiSim or DigiElch or by using freeware that can be downloaded 
from the web such as MECSIM (ref). Now it is the inverse problem of deciding 
which model and combination of parameters best describes the experimental data 
and how good is the fit that usually presents a daunting challenge. Typically, 
the experimentalist who collected the data may elect to “guess” the model that 
is applicable and rely on experience to fit the data by essentially trial and 
error procedures in what almost invariably becomes an extensive series of very 
tedious theory-experiment caparisons. In this heuristic approach, the 
experimentalist decides empirically when an acceptably good fit of data has been 
achieved and then provides a report of the mechanism and parameter values that 
fit the “guessed” mechanism. As an alternative, multi-parameter fitting aided by 
computationally efficient data optimization or more sophisticated approaches 
based on Bayesian inference or other statistical   methods are now available to 
assist with solving the inverse problem.  (refs) Data optimization methodology, 
underpinned by statistics and facilitated by high speed computing, has been 
developed to support many branches of science. Now, as in simulation packages 
developed for the forward problem there are an extensive range of software 
packages available to support complex theory-experiment inverses data 
optimization exercises. Nevertheless, in voltammetry, problems requiring in 
excess of 10 parameters to be estimated are rarely attempted. In particular, if 
the data set available is inadequate as often is the case when using DC cyclic 
voltammetry (inadequate number data points or range of scan rates etc.), then 
the significance of the outcome of a multi-parameter fitting exercise is likely 
to remain equivocal.

In recent work, the Oxford and Monash University Groups have been developing 
protocols to address the issues arising when attempting to parameterize 
increasingly complex mechanisms. In summary, very large data sets containing 
extractable variable time (frequency) domain information are now collected at 
high resolution using instrumentation having 18 bit DAC and ADC converters 
(refs). The waveforms are based on employing a large amplitude periodic waveform 
superimposed onto a cyclic DC potential ramp, so that features related to 
electrochemical impedance spectroscopy and DC cyclic voltammetry are 
simultaneously available as well as additional ones (Refs). Experiment-theory 
comparisons have then been undertaken at levels ranging from fully heuristic to 
multi-parameter fitting in attempts to uniquely define the thermodynamic, 
kinetic, mass transport, capacitance and resistance  related parameters that are 
included in the model used to generate the simulated data. This modelling 
approach uses parameters that have a direct relationship to the physical 
chemistry associated with the reaction mechanisms  unlike the use of equivalent  
circuit models employed traditionally employed in electrochemical impedance 
spectroscopy (EIS), although both approaches are of course mathematically 
interchangeable. 


In the present study, we have taken advantage of access to ever expanding 
computing power as well as software available from many sources to quantify a 
complex mechanism which even in a simplified form contained 15 unknown 
parameters in the model. These parameters were quantified initially by both 
fully heuristic and fully automated data optimization methods. However, a hybrid 
approach in which automated data optimization strategies is informed by 
knowledge gained from the heuristic method of data analysis has been found to 
provide the best fit of theoretical to experimental data. 

Polyoxometalates (POMs) which are of interest in this paper have been widely 
employed in chemistry in diverse fields such as electrocatalysis and 
photocatalysis \cite{symes2013decoupling,lee2012detailed}, sensors 
\cite{wei2014phosphomolybdic,wang2014novel} and capacitors 
\cite{bajwa2013multilayer} and are of interest in many branches of science and 
technology. Since many practical devices based on POMs exploit their extensive 
redox capacity, detailed studies of their electrochemistry are needed to 
facilitate their development.  Due to the widespread importance of POMs but  
inherent complexity of their multistep electron transfer pathways, as expected, 
many qualitative electrochemical studies POMS  by voltammetric methods are 
available,  with quantitative ones almost invariably being  confined to just the 
initial one or two  electron transfer processes. The electrochemistry quantified 
in this study is the reduction of the surface confined polyoxometalate 
\ce{[PMo12O40]3-} This inorganic cluster contains 12 molybdenum atoms in 
oxidation state VI that can be reduced to mixed valent forms in multi-electron 
steps to give highly charged and very basic mixed-valent products that 
facilitate coupling of electron and proton transfer reactions that can occur in 
many combinations. 

The () anion, which is of interest in this paper, in the structural sense is a 
Kegging-type POM (add structural figure) whose voltammetry at solid electrodes 
and polarography at the dropping mercury electrode has been extensively reported 
when dissolved in aqueous electrolyte media, molecular solvents containing 
supporting electrolytes or ionic liquids. In acidic aqueous media,   is known to 
spontaneously adsorb onto electrode surfaces such as glassy carbon 
\cite{choi2009adsorption}, gold \cite{choi2009adsorption}, silver 
\cite{choi2009adsorption}, and reduced graphene oxide 
\cite{ding2014phosphomolybdate}.  DC and AC cyclic voltammograms shown in in 
Figure 1 (to be added) were obtained at a glassy carbon (GC) electrode in 1 M 
H2SO. The three  surface confined reduction processes displayed in these 
voltammograms under  the acidic conditions relevant to this study  and 
designated as Processes I, II and III are each overall  two electron-two proton 
coupled process   that can be represented by  equations 1-3 
\cite{chen2013fabrication}.  

\begin{align}
    \cee{[PMo10O40]3- + 2e- + 2H+ &<=>[E^0,k^0,\alpha, C_{dl}, R_u] 
    H2[PMo12O40]3-} \qquad \text{(Process I)} \label{eq:process_I}\\
    \cee{H2[PMo12O40]3- + 2e- + 2H+ &<=>[E^0,k^0,\alpha, C_{dl}, R_u] 
    H4[PMo12O40]3-} \qquad \text{(Process II)} \\
    \cee{H4[PMo12O40]3- + 2e- + 2H+ &<=>[E^0,k^0,\alpha, C_{dl}, R_u] 
    H6[PMo12O40]3-} \qquad \text{(Process III)}
\end{align}


The peak potentials and are located   at about 350 (Process I), 225 (Process 
II), and 10 (Process III) mV vs \ce{Ag/AgCl} in 1 M \ce{H2SO4} and depend on 
acid concentration (shift of about 60mV per unit change in H2SO4 concentration 
change over the range 0.01 to 1.0M) as expected if a net two- proton transfer 
reaction accompanies a net two-electron transfer reaction. However, extensive 
reduction to give Process IV (not shown) leads to rapid dissolution of solid 
while use of lower acidities facilitates dissolution. Furthermore, describing 
the fully oxidized \ce{[PMo12O40]} as completely unprotonated in 1 M \ce{H1SO4} 
is unlikely to be correct.  

Points for noting gained from perusal of Figure 2 and which should be 
accommodated in simulation-experiment comparisons are that the first two 
processes partially overlap, Process II has a larger peak current magnitude than 
either Processes I or III and the shapes and current magnitudes are not equal 
for all three processes. Importantly, for each of the three reduction steps it 
has been proposed \cite{batchelor2015recent,lopez2014two, 
gonzalez2013reversible,mirceski2016measuring,saveant2006elements,evans2008one} 
that two unresolved reversible one electron transfer steps of purely adsorbed 
material occur as in equations 5 and 6 

\begin{align}
    Ox_{surf} + e^- \cee{&<=>[E^0_1,k^0_1]} I_{surf} \label{eq:ox_to_i} \\
    I_{surf} + e^- \cee{&<=>[E^0_2,k^0_2]} Red_{surf}, \label{eq:i_to_red}
\end{align}

where $Ox_{surf}$ and $Red_{surf}$ are the fully oxidized formally 
\ce{[PMo12O40]3-} (written for convenience in the non-protonated form) and 2 
electron reduced formally protonated \ce{[PMo12O40]5-} species bound to the 
electrode surface while $I_{surf}$ is an intermediate or half-way reduced 
surface confined  protonated \ce{[PMo12O40]4-} species. Even the simplest 
possible level of data analysis with all six electron transfer steps treated as 
reversible and all other parameter values known requires the determination of 
six reversible formal potentials ($E^0$ values). Inclusion of the electrode 
kinetics for requires the addition of six electron transfer rate constants  
($k^0$  values) along with six charge transfer coefficients ($\alpha$ values) if 
the Butler-Volmer relationship is used to model the electron transfer processes.  
Additionally, $C_{dl}$, $R_u$, and the adsorption isotherm need to be modelled 
as in principle do the thermodynamics and kinetics of the chemical (acid-base)   
reactions coupled to electron transfer. The existence of unresolved overall 2 
electron processes precludes any simple interpretation of the thermodynamics and 
electrode kinetics and there are well in excess of thirty parameters present in 
the full POM reduction modeling exercise. As noted above, even at this level of 
complexity, all parameters could be included in the forward problem using 
MECSim, DigiElch or other software packages. However, it is not likely to be 
realistic to simply undertake a 30 or more parameter determination exercise 
using a manual heuristic or even an automated data so the optimization 
experiment –theory fitting exercise and expect to obtain a chemically and 
statically reasonable unique value for each one. In a practical sense it will be 
shown in this study that the best opportunity to approach this goal is by 
integration of both heuristic and automated data optimization approaches to 
achieve   a high level of agreement between experiment and theory for about 15 
parameters that are shown to significantly influence the voltammetry.

Finally, in this Introduction it is emphasized that the data set obtained 
experimentally has to be sufficient in quality and quantity to justify the 
conclusions reached in a parameterization exercise and capable of being mimicked 
by the simulations. AC voltammetry in square wave, sinusoidal or other forms as 
well as EIS provide significant advantages over traditional DC methods in 
quantitative studies of electrode processes 
\cite{bond2015integrated,guo2015fourier}. If large amplitude AC signals are 
employed, the significantly amplified higher-order AC harmonic components become 
available that are virtually free of background capacitance current and are also 
highly sensitive to electrode kinetics. The ability to resolve the aperiodic DC, 
fundamental and higher order harmonics is crucial in heuristic forms of data 
analysis. For simple problems, comparison of experimental data with the 
numerical simulations of appropriate models have provided  good estimates of 
$E^0$,  $k^0$, $\alpha$, $R_u$ and $C_{dl}$ from a single experiment, using  
both heuristic and data optimization forms of experiment -theory comparison.  
Large amplitude AC voltammetry also has been applied to the determination of the 
electrode kinetics of surface-confined enzymes. In some cases, their behavior  
mimics that   with \ce{[PMo12O40]3-} in the sense  that  acid-base chemistry   
can be coupled with an unresolved 2 electron transfer process to provide a 
problem akin to a combination of equations \ref{eq:process_I}, \ref{eq:ox_to_i} 
and \ref{eq:i_to_red}. In this study, experience acquired in parameterization of 
these simpler systems (refs) is crucial in addressing the vastly more complex 
problem of parameterization of the six electron reduction of surface confined 
\ce{[PMo12O40]3-}.

\section{Experimental}

\subsection{Chemicals and Reagents}


Sulfuric acid (\ce{H2SO4}, Merck,) and phosphomolybdic acid (\ce{H3PMo12O40}, 
Sigma-Aldrich, ≥ 99.99\%), were used as received. Water employed for preparation 
of all aqueous solutions was obtained from a Milli-Q water purification system.

\subsection{Instrumentation and procedures}


AC voltammetric measurements were undertaken using in-house instrumentation with 
an applied sine wave of amplitude ($\Delta E$) of 20 mV and frequency ($f$) of 
9.02 or 60.05 Hz being superimposed onto the DC cyclic ramp which had a scan 
rate ($v$).   DC voltammetric data were also obtained with this instrument. A 
conventional three-electrode potentiated system was used to collect experimental 
data at 25°C.  GC (diameter = 3.0 mm, CH Instruments, USA) was used as the 
working electrode,
\ce{Ag/AgCl} (3M \ce{KCl}) as the reference electrode and platinum wire as the 
auxiliary electrode with 1.0M \ce{H2SO4} being the supporting electrolyte.

\subsection{Simulations and data analysis}

Simulations of total current AC voltammograms used in heuristic forms of data 
analysis were carried out with MECSim (Monash Electrochemistry Simulator) 
software. Modelling was undertaken with code specifically written for use in the 
automated data optimization exercises. Details of the model and parametrization 
employed are provided in the Results and Discussion section.  In Fourier 
transformed  AC voltammetry (FTACV) the experimental and simulated total current 
time domain data were converted to the frequency domain to generate the power 
spectrum [ref]. Frequencies corresponding to the AC harmonics and the aperiodic 
DC component were then selected from the power spectrum and then subjected to 
band filtering and inverse Fourier transformation to obtain resolved DC and AC 
components as a function of time. $R_u$ was determined experimentally from the 
$R_u C_{dl}$ time constant at potentials where no faradaic current was present.

\section{Results and Discussion}

\subsection{The Model}\label{sec:model}

To simulate the total current AC voltammetric data, a mathematical model has to 
be employed. To mimic the 3 surface confined processes summarized in equations 2 
to 4, a series of well-known relationships were employed in combination to give 
a total model as follows:

\begin{enumerate}
\item Electron transfer model: the Butler-Volmer relationship was used for each 
    of the six electron transfer steps which requires the introduction of six 
        $E^0$, $k^0$ and $\alpha$ parameters.  Alternately, the mathematically 
        more sophisticated Marcus-Hush relationship could have been used.  
        Neither thermodynamic nor kinetic dispersion are included in the 
        electron transfer model.

\item Acid-base chemistry: proton transfer reactions coupled to electron 
    transfer were assumed to be diffusion controlled and hence reversible and 
        subject to a solely thermodynamic description as is $E^0$. On this 
        basis, the unknown acid ($pK_a$) equilibrium constants have been 
        combined with the $E^0$ values so effectively only six reversible 
        potential ($E_{rev}$) thermodynamically relevant parameters with their 
        six $k_{rev}$ and $\alpha_{rev}$ values need to be parameterized.  
        However, this means that the individual $E^0$, $k^0$, $\alpha$, and 
        $pK_a$ values remain unknown.  

\item Adsorption isotherm: the Langmuir model was used, which means that a 
    surface coverage parameter is needed. Other isotherms are available such as 
        Frumkin or Temkin.

\item Uncompensated resistance: Ohms law used ($R_u$ parameter introduced).

\item Double layer capacitance: Modeling undertaken assumes a simple $R_u$$C_{dl}$ time 
    constant   applies at each potential and that $C_dl$ is independent of 
        potential ($C_{dl}$ parameter needed). In fact $C_{dl}$ depends on potential

\item Numerous other parameters are included in the model (see theory below), 
    but most are known such as AC amplitude DC scan rate, electrode area, 
        temperature etc. However, as a check on fidelity of experimental data, 
        AC frequency and phase angle are treated as parameters to be estimated 
        in the automated but not heuristic data optimization approach to ensure 
        absence of instrumental artefacts.  

\end{enumerate}

In summary, the model chosen for simulation of AC voltammograms requires a total 
of 17 parameters to be estimated by automated data optimization and 15 
heuristically under circumstances where there are imperfections and 
uncertainties in each aspect of the model as noted above 

The sequence of six electron transfers (Equations \ref{eq:ox_to_i} and 
\ref{eq:i_to_red} for Process I, II and II) are modelled by the sequence of 
quasi-reversible reactions

\begin{align} \label{eq:all_reaction}
    A + e^- \underset{k^1_{ox}(t)}{\overset{k^1_{red}(t)}{\rightleftarrows}} B, 
    \\
    B + e^- \underset{k^2_{ox}(t)}{\overset{k^2_{red}(t)}{\rightleftarrows}} C, 
    \\
    C + e^- \underset{k^3_{ox}(t)}{\overset{k^3_{red}(t)}{\rightleftarrows}} D, 
    \\
    D + e^- \underset{k^4_{ox}(t)}{\overset{k^4_{red}(t)}{\rightleftarrows}} E, 
    \\
    E + e^- \underset{k^5_{ox}(t)}{\overset{k^5_{red}(t)}{\rightleftarrows}} F, 
    \\
    F + e^-  \underset{k^6_{ox}(t)}{\overset{k^6_{red}(t)}{\rightleftarrows}} G,
\end{align}

where the forward and backwards reaction rates are given by the Butler-Volmer 
equations

$$ \label{eq:rate1}
k^i_{red}(t) = k^0_i \exp\left(-\frac{\alpha_i F}{RT} [E_r(t) - E^0_i] \right),
$$
$$ \label{eq:rate2}
k^i_{ox}(t) = k^0_i \exp\left((1-\alpha_i)\frac{F}{RT} [E_r(t) - E^0_i] \right).
$$

The reduced voltage $E_r(t)$ is the input voltage $E(t)$ minus the voltage due 
to resistance $R_u$.

$$
E_r(t) = E(t) - R_u I_{tot}(t)
$$

and the input voltage is the sum of dc and ac components

$$
E(t) = E_{dc}(t) + dE\sin(\omega t + \eta)
$$

$$
E_{dc}(t) = \left. \begin{cases} E_{s} + vt, & \text{for } 0 \le t < t_r\\
E_{r} - v(t-t_r), & \text{for } t_r \le t
\end{cases} \right. ,
$$

$$
t_r = \frac{E_{r}-E_{s}}{v}
$$

Let $\bm{\theta} = (\theta_1,\theta_2,\theta_3,\theta_4,\theta_5,\theta_6)$ be 
the proportion of $(A,B,C,D,E,F)$ on the surface of the electrode, where 
conservation of the total gives the proportion of $G$ as 
$1-\theta_1-\theta_2-\theta_3-\theta_4-\theta_5-\theta_6$. 

The ODE governing the behaviour of $\bm{\theta}$ is given by

$$
\label{eq:ode}
\frac{d\bm{\theta}}{dt} = K \bm{\theta} + \mathbf{c}
$$

where the stoichiometry matrix $K$ is given by

$$
K = \begin{pmatrix}
    - k^1_{red} & k^1_{ox} & 0 & 0 & 0 & 0 \\
      k^1_{red} & -k^2_{red}-k^1_{ox} & k^2_{ox} & 0 & 0 & 0 \\
      0 & k^2_{red} & -k^3_{red}-k^2_{ox} & k^3_{ox} & 0 & 0 \\
      0 & 0 & k^3_{red} & -k^4_{red}-k^3_{ox} &  k^4_{ox} & 0 \\
      0 & 0 & 0 & k^4_{red} & -k^5_{red}-k^4_{ox} & k^5_{ox} \\
      -k^6_{ox} & -k^6_{ox} & -k^6_{ox} & -k^6_{ox} & k^5_{red} -k^6_{ox}& 
      -k^6_{red}-k^5_{ox} -k^6_{ox}\\
\end{pmatrix},
$$

and the vector $\mathbf{c}$ arrises due to the $1$ in the proportion of $G$, and 
is given by

$$
\mathbf{c} = \begin{pmatrix}
    0 \\
    0 \\
    0 \\
    0 \\
    0 \\
    k^6_{ox} \\
\end{pmatrix}.
$$

We use a backwards euler discretisation of the time gradent in Eq.~\ref{eq:ode}

$$
\frac{\bm{\theta}^{n+1} - \bm{\theta}^{n}}{\delta t} = K \bm{\theta}^{n+1} +
\mathbf{c}
$$

With the exception that the total current was treated explicity. That is, 
$I_{tot}^n$ was used to calculate $K$. This leads to the following linear system 
to be solved at each timestep

$$
(I - \delta t K)\bm{\theta}^{n+1} = \bm{\theta}^n + \delta t \mathbf{c}
$$

where $\delta t$ is the timestep, and $\bm{\theta}^n$ is the proportion vector 
at timestep $n$.


The total current measured is the sum of the capacitive and faridaic components

$$
I_{tot} = I_c + I_f,
$$

which are given by

$$
I_c = C_{dl} \left(1 + C_{dl1} E_r(t) + C_{dl2} E_r^2(t) + C_{dl3} 
E_r^3(t)\right) \frac{dE_r}{dt},
$$

$$
I_f = F a \Gamma \frac{de}{dt}.
$$

where the change of charge with time is given by

$$
\frac{de}{dt} = 
\begin{pmatrix}
    -k^1_{red}-k^6_{ox} \\
    k^1_{ox}-k^2_{red}-k^6_{ox} \\
    k^2_{ox}-k^3_{red}-k^6_{ox} \\
    k^3_{ox}-k^4_{red}-k^6_{ox} \\
    k^4_{ox}-k^5_{red}-k^6_{ox} \\
    k^5_{ox}-k^6_{red}-k^6_{ox}
\end{pmatrix}^T \bm{\theta} +  k^6_{ox} 
$$


\subsection{Heuristic Method of Parameter Estimation}

The heuristic or exclusively experimenter based trial and error method to be 
tractable requires simplifying the problem and obtaining initial estimates of 
parameters in a sensible stepwise fashion by relying on experience gained with 
simpler problems. In this exercise, the solution to the forward problem was 
obtained from the simulation package MECSim with an initial guess made for each 
parameter. Simulations with new sets of parameters were then iteratively changed 
in the direction dictated by the experience of the experimenter until this 
individual decided an acceptably “good fit” to data had been achieved. Of course 
a different experimentalist may decide that the “good fit” is achieved with a 
different set of parameters!

The initial scrutiny of data involved visual interrogation of   20mV amplitude 
FTAC data sets at 9 and 60 Hz (Figs x and y), their power spectra and the first 
4 AC harmonics (Figs e, f) and a dc cyclic voltammogram (figure a). This 
overview allowed initial guesses or estimates to be made for some parameters 
along with some deductions as to the validity of the model as follows:

(a) It was noted in the fundamental AC harmonic that a potential (time) region 
exists prior to onset of faradaic current for process I that is purely 
capacitive.  This region was analyzed in terms of the $R_u  C_{dl}$ time constant to 
give an estimate of $R_u$ as 49 +/- 5 Ohm. 

(b) By matching experiment and theory  for the same potential (time) region in 
the fundamental  harmonic that is devoid of faradaic current a described in (a) 
and effectively use of equation x (to be added), $C_{dl}$ was estimated to be 7 
+/- 1 (TODO: units) F cm-2. However, the sloping baseline in  the DC cyclic 
voltammogram and mismatch of positive and negative potential regions of 
fundamental harmonic devoid of faradaic current, and indeed the aperiodic DC 
component of FTACV data, it is clear that $C_{dl}$  potential dependent, rather 
than potential independent, as assumed in the model. Thus, treating $C_{dl}$ as a 
potential independent parameter in an experiment versus theory exercise will 
represent an imperfect approximation.  More complex potential dependent 
estimates of $C_{dl}$ could have been included in the model. However, perusal of 
third and fourth AC harmonics (Figure x) reveals no background current in these 
responses, implying that the electrode kinetics can be estimated from these data 
that are devoid if capacitance current.  Consequently, moving to the more 
complex capacitance model was not regarded as necessary, particularly under 
conditions where the Ohmic drop or $I_{total} R_u$ is small.

(c) The surface coverage was found to be $\Gamma$ = 50 +/- 5 pmole cm-2 based on 
integration of DC current time-data to give the charge associated with reduction 
of the POM and use of Faraday’s Law (equation y to be added). The charge used in 
the estimation of $\Gamma$ was calculated from the sum of that derived from 
processes I and II (n=4) or process   III (n = 2) after   background 
(capacitance current) correction of a DC linear sweep voltammogram (ref)

(d) Since all three processes are known to be overall two electron, are sharp 
and give a series of well-defined higher order AC components, the $k_0$ values were 
assumed to be large. The initial guess was that all are sufficiently large and 
that with a low frequency of 9Hz that all one electron processes could initially 
be assumed to be reversible.  Furthermore, the two $E_{rev}$ values associated with 
each of processes I, II and III are clearly not well separated in the direction 
that the second electron transfer step is more negative than the second.  Thus, 
the reversible potentials of the two  overlapping  steps in all cases were 
concluded to be either almost the same  or the second electron transfer step was 
less negative  than the first (crossed over potentials because of the 
thermodynamic contribution of protonation step coupled to electron transfer as 
noted above) To obtain more detail, Figure 2 was generated to provide simulated 
fundamental and third harmonic AC  data where the separation of the two $E_{rev}$ 
values in process was II fixed at zero (largest current magnitude ) while  
processes  II and III had variable separations in the $E_{rev}$ ($\Delta E_{rev}$) 
values from 0 to 55 mV with all $k^0$ and $\alpha$  values being set at 1 x 10 8  
s-1  and 0.50 respectively to mimic a reversible process  and other parameters 
set at estimates approximating those deduced in (a), (b) and (c) above.  Greater 
crossover potentials   than 53 mV were deemed unreasonable. By eye, and with 
some fine tuning, $\Delta E_{rev} = 30$, 0 and 27  for process I, II and III 
respectively gave match of theory and experiment for the 9Hz low frequency data 
set. To confirm that the assumption of reversibility is valid and what this 
means, Figure 3  was generated with  simulated fundamental and third harmonic 
response for the pairs of $k_0$ values always the same but ranging from $1.0 \times 
10^3$ to $10^8$  s-1  with $\alpha$ being set at 0.50 and $\Delta E_{rev} = 33$, 
0 and 45.  At the low frequency of 9 Hz, all three processes are insensitive to 
$k_0$ electrode kinetics in excess of $10^2$ s-1. Thus it was concluded from 
this trial and error analysis that $\Delta E_{rev} = 30$, 0 and 27 were 
appropriate for processes I, I and III which translates to $E_{rev}$ values of 
368, 338, 227, 227 11 and -16 mV Ag/AgCl and that some if not all $k_0$ values are 
larger than about $10^2$ s-1.   

(e) Examination of the higher frequency 60 Hz data set implied that the 
electrode kinetics could possibly be determined from theory experiment 
comparisons of the higher order AC harmonics However, the only reasonable trial 
and error heuristic approach available was to set the $k_0$ value for each pair 
of processes to be equal with $\alpha = 0.50$, assume $E_{rev}$ values and other 
parameters are known from 9 Hz data or can be re-estimated as for 9 Hz data. Now 
only  pairs of ko values are the along with the surface coverage scaling factor 
$\Gamma$ remain as unknowns. After much tedious trial and error examining many 
simulations, the reasonably good match of theory and experimental data shown in 
figure x was obtained for the total current data and in figure y for AC 
harmonics -1 to 4 when the parameters summarized in Table 1 and in captions to 
these Figures are used.  Never the less, the significance of the absolute ko 
values deduced from this exercise and included in Table 1 remains unclear, apart 
from implying  that fast electrode kinetics  approaching the reversible limit is 
evident for each of processes I, II and III

Since the fit obtained heuristically is imperfect and to make tractable had to 
assume that pairs of $k_0$ values are equal, it now needs to be established as what 
outcome could be achieved   via multi-parameter data optimization and how this 
exercises should be implemented since almost certainly simply seeking to obtain 
15 or more unknown parameters from a single multi-parameter fit would be futile.  
Indeed, it will emerge from this exercise that a hybrid approach based using 
knowledge obtained above heuristically, supported by computationally efficient 
data optimization are needed to obtain chemically sensible outcomes.

\begin{figure}[htbp]
%\includegraphics[width=\textwidth]{images/ac_volt.pdf}
    \caption{\it{Simulated fundamental and third harmonic FTAC voltammograms 
    used to assist in heuristic form of data analysis for the six electron POM 
    reduction process with designated $E_{rev}$ (mV) values. Simulations 
    employed $f = 9.0$ Hz, $\Delta E = 20$ mV, $v = 113 mV.s^{-1}$, $E_{s} = 
    600$ mV, $E_{r} = -160$ mV, $R_u = 50 \Omega$, $C_{dl} = 6 \mu 
    F.cm^{-2}$, $T = 298.2$ K, $A = 7.86 \times 10^{-3} cm^2$, and $\Gamma = 53 
    pmole.cm^{-2}$, all $k_0 = 1 \times 10^8 s^{-1}$ and $\alpha =0.50$, and 
    designated $\Delta E_0$ values for first and third processes with fixed 
    value of  $\Delta E_{rev} = 0$ mV for second process.}}
    \label{fig:sim_for_heuristic_E}
\end{figure}


\begin{figure}[htbp]
%\includegraphics[width=\textwidth]{images/ac_volt.pdf}
    \caption{\it{Simulated fundamental and third harmonic FTAC voltammograms 
    used to assist in heuristic form of data analysis for the overall six 
    electron reduction process with designated pairs of $k_{rev} (s^{-1})$ 
    values for the six-electron POM reduction processes. Simulations employed $f 
    = 9.0$ Hz, $\Delta E = 20$ mV, $v = 113 mV s^{-1}$, $E_{s} = 600$ mV, 
    $E_{r} = -160$ mV, $R_u = 50 \Omega$, $C_{dl} = 6 \mu F.cm^{-2}$, $T = 
    298.2$ K, $A = 7.86 \times 10^{-3} cm^2$, and $\Gamma = 53 pmole.cm^{-2}$ 
    and $\Delta E_{rev} = 33$, 0, 45 mV for first to third processes, 
    respectively.}}
    \label{fig:sim_for_heuristic_k}
\end{figure}

\begin{figure}[htbp]
\includegraphics[width=\textwidth]{figs/heuristic_visHarmonics.pdf}
    \caption{\it{Comparison of simulated and experimental first to fourth 
    harmonic FTAC voltammograms obtained for first three reductions process of 
    \ce{[PMo12O40]3-} adsorbed on a GC in contact with 1.0 M \ce{H2SO4}. AC
    experimental data obtained with $f = 60$ Hz, $\Delta E = 20$ mV. Simulated
    data obtained with $R_u = 50 \Omega$, $C_{dl} = 8 \mu F.cm^{-2}$, $T =
    298.2$ K, $A = 7.07 \times 10^{-3} cm^2$,
    $(E^0_1,E^0_2,E^0_3,E^0_4,E^0_5,E^0_6) = (0.368, 0.338, 0.227, 0.227, 0.011,
    -0.016)$ V, $(k^0_1,k^0_2,k^0_3,k^0_4,k^0_5,k^0_6) = (7300, 7300, 10^4,
    10^4, 2500, 2500)\text{ s}^{-1}$,and $\Gamma = 0.7*53 (TODO why 0.7 needed?)
    pmole.cm^{-2}$ and other parameters given in tables and text.}}
    \label{fig:sim_and_exp}
\end{figure}


\section{Parameter Estimation using Automated Data Optimization Methods}\label{sec:automated}

The set of parameters to be found is given by the vector $\mathbf{p}$

$$
\mathbf{p} =
(E^0_1,E^0_2,E^0_3,E^0_4,E^0_5,E^0_6,k^0_1,k^0_2,k^0_3,k^0_4,k^0_5,k^0_6,\Gamma).
$$

We also have an experimental current trace given by a vector $\mathbf{I}^{exp}$,
where each element of the vector $I^{exp}_i$ is the measured current at sample
time $t_i$. Using the simulated model given in Section \ref{sec:model}, we
can calculate an equivilent simulated current trace
$\mathbf{I}^{sim}(\mathbf{p})$, with elements $I^{sim}_i$ calculated at the same
set of sample times $t_i$. Since the timestep $\delta t$ is unlikely to match
the experimental sample times, we use linear interpolation to resample the
simulated current trace to the experimental sample times.

A traditional optimisation function for calculating the optimal parameters given
$\mathbf{I}^{exp}$ is the 2-norm distance metric

\begin{equation}
    \mathcal{F}(\mathbf{p}) = ||\mathbf{I}^{exp}-\mathbf{I}^{sim}(p)|| = \sqrt{\frac{1}{N}\sum_{i=0}^{N} [I^{exp}_i-I^{sim}_i(p)]^2}
\end{equation}

Minimising this distance will give an optimal set of parameters. We perform the
minimisation using the genetic algorithm CMA-ES (TODO cite), using the
following parameter bounds:

\begin{equation}\label{eq:bounds}
    \begin{pmatrix}
        E_{r}+\Delta E \\
        E_{r}+\Delta E \\
        E_{r}+\Delta E \\
        E_{r}+\Delta E \\
        E_{r}+\Delta E \\
        E_{r}+\Delta E \\
        0 \\
        0 \\
        0 \\
        0 \\
        0 \\
        0 \\
        0.1
    \end{pmatrix}^T
    \le \mathbf{p} \le
    \begin{pmatrix}
        E_{s}-\Delta E \\
        E_{s}-\Delta E \\
        E_{s}-\Delta E \\
        E_{s}-\Delta E \\
        E_{s}-\Delta E \\
        E_{s}-\Delta E \\
        10^4 \\
        10^4 \\
        10^4 \\
        10^4 \\
        10^4 \\
        10^4 \\
        5
    \end{pmatrix}^T
\end{equation}

For this problem, some sort of regularisation is neccessary to ensure
convergence to the optimal parameter set. We find in practice that using the
standard 2-norm distance metric often results in the algorithm stalling in
parameter regions where the values of the reversible potential $E^0_i$ for the six
different reactions have swapped their order. That is, for example, $E^0_1 >
E^0_3$, even though we know that Process I occurs before Process II. 

We regularise the problem by using a minimisation function inspired by the
statistical Bayesian framework. Assuming independent Gaussian measurement noise
at every timestep, we can write down the log-likelihood of $\mathbf{I}^{exp}$
occuring given a parameter set $\mathbf{p}$ as

\begin{equation}
    \mathcal{L}(\mathbf{p}) = \sqrt{\frac{1}{2 \sigma^2} \sum_{i=0}^{N} [I^{exp}_i-I^{sim}_i(p)]^2}
\end{equation}

where $\sigma$ is an unknown measurement noise level that we add to the
parameter vector $\mathbf{p}$.

We encode our knowlege of the parameters in a Bayesian normal prior for each of
the $E^0_i$ parameters with a mean $m_i$ and standard deviation $\sigma_i$. 

\begin{equation}
    \mathcal{P}_i = \frac{1}{\sqrt{2 \pi \sigma_i^2}} e^{-\frac{(p_i-m_i)^2}{2 \sigma_i^2}} \text{      for } i \le 6
\end{equation}

The mean encodes our "best guess" for the value of $E^0_i$ and the standard
deviation encodes how confident we are in this "best guess". For the remaining
paramters we use a uniform prior between the lower and upper bounds given in
Eq.~\ref{eq:bounds}. Since these uniform priors are independent of the parameter vector $\mathbf{p}$ they do not contribute to the minimisation. 

Given the large number of experimental data samples provided ($N = XXX$), there
is a danger that the log-likelihood function will dominate over the prior (i.e.
the prior will have no effect on the performance of the algorithm), therefore we
also scale the log-likelihood based on $N$. Combinging the prior with the scaled
log-likelihood gives the proposed objective function to be minimised

\begin{equation}\label{eq:objective}
    \mathcal{F}(\mathbf{p}) =  \frac{\mathcal{L}(\mathbf{p})}{N} + 
    \sum_{i=1}^6 \log \mathcal{P}_i
\end{equation}

Using CMA-ES to minimise Eq.~\ref{eq:objective} and using
$(m_1,m_2,m_3,m_4,m_5,m_6) = ()$ and $\sigma_i = (\sigma_0+1)(E_s-E_r)$, where
$\sigma_0$ is allowed to vary between $0 \le \sigma_0 \le 20$ gives the results
shown in Figure~\ref{fig:quasi_results}. For each value of $\sigma_0$, 20 separate
runs of the CMA-ES algorithm were done. The mimimum $\mathcal{F}$ scores for
each were compared, and the number of "good fits" were calculated, where a good
fit is determined as one whose $\mathcal{F}$ score was within 1\% of the lowest
$\mathcal{F}$ for those 20 runs. This information, along with the individual
scores for all the attempts, is displayed in Figure~\ref{fig:quasi_results}. The
results show that a $\sigma_0 < 0.2$ is required for an 80\% reliability for the
automated fitting. These results are very promising, since this indicates that
an initial guess on each $E^0$ value only needs to be accurate within 20\% of
the potential sweep range for automated fitting to be reliable. An experiance
voltammetry user can easily read this estimate off a simple current-voltage plot
of the experimental trace.

\begin{figure}[htbp]
\includegraphics[width=\textwidth]{figs/quasireversible.pdf}
    \caption{\it{Results of automated fitting of model to experimental data
    using quasireversible model. The left vertical axis shows the objective
    function Eq.~\ref{eq:objective} versus the scaled standard deviation
    $\sigma_0$ of the normal prior on the reversible potentials (i.e. the range
    over which the user expects the optimal $E^0_i$ to be found). A higher
    $\sigma_0$ indicates a wider range, or less confidence in the initial guess.
    The right vertical axis shows the number of good fits out of 20 attempts,
    where a good fit is defined as one that results in an minimal $\mathcal{F}$
    within 1\% of the best of the 20 attempts. The results show that a range
    smaller than 0.2 times the range of the potential sweep is required for
    automated fitting with an reliability greater than 50\%.}}
    \label{fig:quasi_results}
\end{figure}

Figure~\ref{fig:quasi_best_result} show a comparison of the best fit with
$\sigma_0=0.2$, against the experimental data. The first six harmonics are shown
for comparison, and it can be seen visually that the fit due to the automated
process is excellent. Hence we have shown that an automated fitting process can
be reliable, given relativly weak prior information by the user, and result in a
fit that is equal or superior to the overly laborous, heuristic method.

\begin{figure}[htbp]
\includegraphics[width=\textwidth]{figs/quasi_visHarmonics2_64_629248.pdf}
    \caption{\it{Comparison of simulated and experimental results for the
    automated fitting algorithm presented in Section~\ref{sec:automated}. The
    first to sixth harmonic FTAC voltammograms obtained for first three
    reductions process of \ce{[PMo12O40]3-} adsorbed on a GC in contact with 1.0
    M \ce{H2SO4}. AC experimental data obtained with $f = 60$ Hz, $\Delta E =
    20$ mV. Simulated data obtained with $R_u = 50 \Omega$, $C_{dl} = 8 \mu
    F.cm^{-2}$, $T = 298.2$ K, $A = 7.07 \times 10^{-3} cm^2$,
    $(E^0_1,E^0_2,E^0_3,E^0_4,E^0_5,E^0_6) = (0.363, 0.348, 0.215, 0.239, 0.002,
    -0.018)$ V, $(k^0_1,k^0_2,k^0_3,k^0_4,k^0_5,k^0_6) = (3650, 5069, 6698, 9981,
    4428, 5766)\text{ s}^{-1}$, and $\Gamma = 30.05\text{ pmole.cm}^{-2}$.}}
    \label{fig:quasi_best_result}
\end{figure}

The high reaction rate $k^0$ values found by both the heuristic and automated
fitting process, indicate that all the reactions are operating close to the
reversible limit. Therefore it is instructive to consider fitting an easier
model with each $k^0_i$ parameters set to a constant value of $10^4 s^{-1}$
(increasing $k^0_i$ above this value does not change the results significantly
since the simulated model is in the reversible limit). Similar results comparing
the number of good fits with varying $\sigma_0$ are shown in
Figure~\ref{fig:rev_results}. Here we see that the prior information neccessary
for the reversible potential parameters is much the same as for the
quasireversible case, even though we are fitting a far lower number of
parameters (7 instead of 13). This indicates that it is non-linear response of
the system in response to changes in $E^0_i$ that causes difficulty in the
mimimisation process, rather than the high dimensional parameter space.

\begin{figure}[htbp]
\includegraphics[width=\textwidth]{figs/reversible.pdf}
    \caption{\it{Results of automated fitting of model to experimental data
    using the reversible model (i.e. $k^0_i = 10^{4}\text{ s}^{-1}$). The left
    vertical axis shows the objective function Eq.~\ref{eq:objective} versus the
    scaled standard deviation $\sigma_0$ of the normal prior on the reversible
    potentials (i.e. the range over which the user expects the optimal $E^0_i$
    to be found). A higher $\sigma_0$ indicates a wider range, or less
    confidence in the initial guess. The right vertical axis shows the number of
    good fits out of 20 attempts, where a good fit is defined as one that
    results in an minimal $\mathcal{F}$ within 1\% of the best of the 20
    attempts. The results show that a range smaller than 0.2 times the range of
    the potential sweep is required for automated fitting with an reliability 
    greater than 50\%.}}
    \label{fig:rev_results}
\end{figure}

We also provide the best fit for the reversible case in
Figure~\ref{fig:rev_best_result}. This is, as expected, virtually
indestinguishable from the quasireversible case, indicating that the reactions
are proceeding close to the reversible limit.

\begin{figure}[htbp]
\includegraphics[width=\textwidth]{figs/rev_visHarmonics6_17_810104.pdf}
    \caption{\it{Comparison of simulated and experimental results for the
    automated fitting algorithm presented in Section~\ref{sec:automated}, and
    for the reversible model (i.e. $k^0_i = 10^{4}\text{ s}^{-1}$). The first to
    sixth harmonic FTAC voltammograms obtained for first three reductions
    process of \ce{[PMo12O40]3-} adsorbed on a GC in contact with 1.0 M
    \ce{H2SO4}. AC experimental data obtained with $f = 60$ Hz, $\Delta E = 20$
    mV. Simulated data obtained with $R_u = 50 \Omega$, $C_{dl} = 8 \mu
    F.cm^{-2}$, $T = 298.2$ K, $A = 7.07 \times 10^{-3} cm^2$, and
    $(E^0_1,E^0_2,E^0_3,E^0_4,E^0_5,E^0_6) = (0.365, 0.348, 0.215, 0.239, 0.002,
    -0.019)\text{ V}$, and $\Gamma = 30.01\text{ pmole.cm}^{-2}$.}}
    \label{fig:rev_best_result}
\end{figure}




\section{Conclusions --- still to be written}

\section{Acknowledgements}

K.O. acknowledges the award of Endeavour Research Fellowship for financial
support that enabled him to undertake this study at Monash University. The
authors also would like to thank Miss Sze-yin Tan for helpful discussions on
undertaking simulations using MECSim and the Australian research Council for
financial assistance in the form of a Discovery Program grant


\bibliography{poms.bib}

\end{document}
